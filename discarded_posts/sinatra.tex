\begin{figure}[htbp]
\centering
\includegraphics{http://ghchart.rshah.org/2016rshah}
\caption{2016rshah's Github Contributions Chart}
\end{figure}

So recently I created this little Ruby on Rails app that lets you embed
your Github Contribution calendar into any HTML (or Markdown) as an
image. You can see an example of mine above which was inserted into HTML
with
\texttt{\textless{}img\ src="http://ghchart.rshah.org/2016rshah"\ alt="2016rshah\textquotesingle{}s\ Github\ Contributions\ Chart"\ /\textgreater{}}.
Of course any github user can easily make their own by replacing their
username. You can find more information on
\href{https://github.com/2016rshah/githubchart-api}{github} or try it
out \href{http://ghchart.rshah.org/}{on the site}.

It is simple, straightforward, but still pretty neat, if I do say so
myself. But soon I decided to rewrite the application in Sinatra, and
here's why.

\subsection{Starting with Rails}\label{starting-with-rails}

When I started this project, I dove in with rails because I had used it
before on apps like \href{}{a CNN article generator} and a
\href{}{bookshelf}. After I had my basic functionality down, the file
structure looked like a typical rails app. For reference, here is the
\href{https://github.com/2016rshah/githubchart-api/tree/bb813c46188d24a7eb620b69c2a4f07baacc505f}{file
structure at that point} (created using
\href{http://mama.indstate.edu/users/ice/tree/}{tree}).

\begin{verbatim}
.
├── Gemfile
├── Gemfile.lock
├── README.rdoc
├── Rakefile
├── app
│   ├── assets
│   │   ├── images
│   │   ├── javascripts
│   │   │   ├── application.js
│   │   │   └── chart.coffee
│   │   └── stylesheets
│   │       ├── application.css
│   │       ├── chart.scss
│   │       └── scaffolds.scss
│   ├── controllers
│   │   ├── application_controller.rb
│   │   ├── chart_controller.rb
│   │   └── concerns
│   ├── helpers
│   │   ├── application_helper.rb
│   │   └── chart_helper.rb
│   ├── mailers
│   ├── models
│   │   └── concerns
│   └── views
│       └── layouts
│           └── application.html.erb
├── bin
│   ├── bundle
│   ├── rails
│   ├── rake
│   ├── setup
│   └── spring
├── config
│   ├── application.rb
│   ├── boot.rb
│   ├── database.yml
│   ├── environment.rb
│   ├── environments
│   │   ├── development.rb
│   │   ├── production.rb
│   │   └── test.rb
│   ├── initializers
│   │   ├── assets.rb
│   │   ├── backtrace_silencers.rb
│   │   ├── cookies_serializer.rb
│   │   ├── filter_parameter_logging.rb
│   │   ├── inflections.rb
│   │   ├── mime_types.rb
│   │   ├── session_store.rb
│   │   └── wrap_parameters.rb
│   ├── locales
│   │   └── en.yml
│   ├── routes.rb
│   └── secrets.yml
├── config.ru
├── db
│   ├── schema.rb
│   └── seeds.rb
├── lib
│   ├── assets
│   └── tasks
├── log
├── public
│   ├── 404.html
│   ├── 422.html
│   ├── 500.html
│   ├── favicon.ico
│   └── robots.txt
├── test
│   ├── controllers
│   │   └── chart_controller_test.rb
│   ├── fixtures
│   ├── helpers
│   ├── integration
│   ├── mailers
│   ├── models
│   └── test_helper.rb
└── vendor
    └── assets
        ├── javascripts
        └── stylesheets

35 directories, 47 files
\end{verbatim}

What a pain, right? Especially for the fact that the entirety of my code
could be boiled down to a few lines:

\begin{Shaded}
\begin{Highlighting}[]
\CommentTok{#config/routes.rb}
\NormalTok{get }\StringTok{"/:username"} \NormalTok{=> }\StringTok{"chart#generate"}\NormalTok{, format: }\StringTok{:svg}


\CommentTok{#config/initializers/mime_types.rb}
\DataTypeTok{Mime}\NormalTok{::}\DataTypeTok{Type}\NormalTok{.register }\StringTok{"image/svg+xml"}\NormalTok{, }\StringTok{:svg}  


\CommentTok{#app/controllers/chart_controller.rb}
\KeywordTok{class} \DataTypeTok{ChartController} \NormalTok{< }\DataTypeTok{ApplicationController}
    \KeywordTok{def} \NormalTok{generate}
        \NormalTok{svg = }\DataTypeTok{GithubChart}\NormalTok{.new(}\StringTok{user: }\NormalTok{params[}\StringTok{"username"}\NormalTok{]).svg}
        \NormalTok{respond_to }\KeywordTok{do} \NormalTok{|format|}
            \NormalTok{format.svg \{ render }\StringTok{inline: }\NormalTok{svg\}}
        \KeywordTok{end} 
  \KeywordTok{end}
\KeywordTok{end}
\end{Highlighting}
\end{Shaded}

\subsection{Enter Sinatra}\label{enter-sinatra}

I wasn't happy with how heavy everything felt, so I asked my friends
over at \href{https://www.facebook.com/groups/HHRuby/}{HH Ruby} for a
lightweight alternative to Rails. Basically, they showed me a whole new
world (namely, \href{http://www.sinatrarb.com/}{Sinatra}).

\begin{figure}[htbp]
\centering
\includegraphics{https://33.media.tumblr.com/ede03f400cc31e408b37c82ab5bcdd17/tumblr_inline_ntv7x6qDKx1t2rgkb_500.gif}
\caption{I can show you the world}
\end{figure}

After spending about an hour and a half learning* Sinatra and
refactoring, I emerged victorious!

\begin{verbatim}
.
├── Gemfile
├── Gemfile.lock
├── README.md
├── app.rb
├── config.ru
└── public
    ├── index.css
    └── index.html

1 directory, 7 files
\end{verbatim}

Now, the entirety of my \texttt{app.rb} can be contained more or less to
20 lines with whitespace:

\begin{Shaded}
\begin{Highlighting}[]
\CommentTok{#app.rb}
\NormalTok{require }\StringTok{'sinatra'}
\NormalTok{require }\StringTok{'githubchart'}

\NormalTok{get }\StringTok{'/'} \KeywordTok{do}
    \NormalTok{send_file }\DataTypeTok{File}\NormalTok{.join(settings.public_folder, }\StringTok{'index.html'}\NormalTok{)}
\KeywordTok{end}


\NormalTok{get }\StringTok{'/:username'} \KeywordTok{do}
    \NormalTok{headers }\StringTok{'Content-Type'} \NormalTok{=> }\StringTok{"image/svg+xml"}

    \NormalTok{username = params[}\StringTok{:username}\NormalTok{].chomp(}\StringTok{'.svg'}\NormalTok{) }\CommentTok{#Chomp off the .svg extension to be backwards compatible}

    \NormalTok{svg = }\DataTypeTok{GithubChart}\NormalTok{.new(}\StringTok{user: }\NormalTok{username).svg}

    \NormalTok{stream }\KeywordTok{do} \NormalTok{|out|}
      \NormalTok{out << svg}
    \KeywordTok{end}
\KeywordTok{end}
\end{Highlighting}
\end{Shaded}

Fancy, huh?

\subsection{Conclusion}\label{conclusion}

Look, I'm not saying you should use Sinatra for every enterprise
application you and your team are going to be scaling up for the next
five years. I'm just saying that if you have a small hobby application
and you just want some routes and their corresponding Ruby code, look
into Sinatra. If you want to skip the whole MVC shizam and don't need a
database, Sinatra might be for you.

Also, if you're interested in the application or want your chart, you
can find more information
\href{https://github.com/2016rshah/githubchart-api}{on the github page}
or \href{http://ghchart.rshah.org/}{on the site}.

\paragraph{Footnotes}\label{footnotes}

*Not really learning, because it was essentially the same as
Express+Node.js and everything basically just worked.
